\oldpage[VII]
To the reader that is not familiar with the questions processed in this work, 
it can be useful to give some indications on how to read this one.

It is indispensible to read the \sref[I]{section:I.1} and \sref[I]{section:I.2}, and the \sref[II]{section:II.1} and \sref[II]{section:II.2}. 
We can then read the \sref[II]{section:II.3.1} (flabby sheaves), then the \sref[II]{section:II.4.1} to \sref[II]{section:II.4.4} in order to know the definition and the principal properties of group of cohomology with values in a sheaf. 
It is then indispensible to read the \sref[I]{section:I.4} (spectral sequences) as well as what concerns the soft sheaves at \sref[II]{section:II.3}; 
we can then finish the lecture of \sref[II]{section:II.4} (Essentially are the \sref[II]{section:II.4.5}) to \sref[II]{section:II.4.10}).

Before approaching to the \v{C}ech cohomology, it is useful to read the beginning of \sref[I]{section:I.3}, especially the \sref[I]{section:I.3.1} to \sref[I]{section:I.3.5}; 
we can then, in a first lecture of \sref[II]{section:II.5}, limit oneself to \sref[II]{II.5.1}, \sref[II]{II.5.3}, \sref[II]{II.5.4}, \sref[II]{II.5.7}, \sref[II]{II.5.8}, \sref[II]{II.5.9}, \sref[II]{II.5.10}, by neglecting what concerns the support families. 
Likewise, it is useful to simultaneous read the \sref[I]{section:I.5} and \sref[II]{section:II.7}, by trying, as an exercise, to place oneself in any Abelian category (but having sufficient injective objects). 
It is also useful to construct the purely functorial proof of the fundamental theorem of \sref[II]{section:II.4}.

The reader desire to know some applications of the sheaf theory, where deepen some aspects, will be able to consult with the article of ~Serre on the coherent algebraic sheaves (~Annuals of Math., 61 (1995), pp. 197-278), 
an article of ~Grothendieck appeared at ~Tohoku Math. Journal, and finally the volume of Seminar of the l'E.N.S. of ~H. Cartan dedicated to the several variable complex functions. 
\oldpage[VIII]
We will also be able to consult with the recent volume of ~F.~Hirzebruch in the well-known series of Ergebnisse der Mathematik, which contains the important applications of the sheaves to the theory of compact analytic varieties. 

Finally note that a reference such that Theorem 4.9.3 indicates that we must refer to \sref[II]{II.4.9}, therefore at \sref[II]{section:II.4}, \emph{in the same chapter}, unless mention expressly otherwise 
(use as a result at \sref[II] or the definition contained in the \sref[I]).