\oldpage[I]
Of all the idea that circulate in the mathematical circles, 
that of the public a reference work devoted to the sheaf theory is certainly one of the least original: 
Most specialists in this theory intended to do so at one time or another, 
and we know several publications which deal with the problem
-- in general renoeotyped as if thery were subversive leaflets -- 
the most famous rightly being owned to ~Henri ~Cartan.

Yet we sought in vain to a complete expose, giving all the proves of all the theorems, in a whole.
While the technic of sheaf pervade the most diverse branches of Mathematics, 
such a situation can no longer be tolerated by the technicians: 
this is why a specialist of functional analysis today presents a complete, 
i.e. less incompete than the others, the sheaf theory -- with a vengeance.

It is obvious that a book would be completely useless if it addressed only to specialists of the sheaf theory, or in the same time, 
according to the extra-topologic application of this theory, which assummed the reader were completed informed the classical technic of algebraic topology. 
Therefore we have sought write a book that suppose \emph{no} knowledge of algebraic topology, 
and consequently preceded the presentation of the sheaf thoery with a chapter of the homological algebra, 
which we hope, it will be useful to a certain categories of readers.

In addition to the usual consideration of the exact sequences, functors, complexes, etc ..., 
this first chapter treat essentially three important questions: 
the theory of spectral sequences \sref[I]{section:I.4}, 
the functors $\Ext$ and $\Tor$ \sref[I]{section:I.5}, 
and the simplicial complex \sref[I]{section:I.3}.
Our exposition is obvious shorter and less complete than the one we have found in the recent book of 
~H. ~Cartan and ~S. ~Eilenberg, which concerns the spectral sequences of the $\Ext$ and $\Tor$. 

\oldpage[II]
As for the ``simplicial complex'', they are complexes of chains (or cochains) in which we have ``face operator'' to allow formally perform the classic simplicial calculus; 
a situation that is found not only in the classical thoery of polyhedrons, but also in singular homology, in \v{C}ech cohomology, and in the sheaf theory. 
As ~Kan's recent work also seems to prove that these complexes constitute the natural domain of validatiy of the complete homotopy theory, 
we can claim that the general notion of simplicial complex (mainly due to ~Eilenberg and ~Zilber) is called to play an essential role in algebraic topology.

In particular, we will find in the paragraph 3 an almost complete presentation of the product theory (Cartesian product and cup-product), 
an exposition whose originality is probably to be printed. 
We don't think we should insert in this paragraph a statement of ~Steenrod operator; 
It will be found in the second volume of this book, when we have at our disposal the necessary technics (cohomology of groups, group operater space, singular homology).

The sheaf theory itself occupies the second half of this book.
For readers who are aware of the previously published presentations. 
We will give some indications on the methods we use, since they differ quite significantly from the methods already known, 
and generally go further for the remainder.

After two general paragraphes on the sheaves of sets and the sheaves of modules, we address in paragraph 3 the central problem of the sheaf theory: 
the ``extension'' or the ``lift'' of sections of a sheaf. The essential notion of this point of view, because of its simplicity and usefulness, seems to be that of \emph{flabby} sheaf: 
a sheaf $\sh{F}$ on a space $X$ is flabby if all section of $\sh{F}$ above an \emph{open} set of $X$ can be extended to the whole $X$. 
All sheaf $\sh{F}$ can be extended to a flabby sheaf (For example, the sheaf of germs of not necessary continuous sections in $\sh{F}$); 

Moreover, if we have an exact sequence
\[
    0\rightarrow \sh{F}^0 \rightarrow \sh{F}^1 \rightarrow \ldots
\]
of flabby sheaf of Abelian groups, then the \emph{sections} of this sheaves above any open sets still formed an exact sequence. 
In the paracompact space, it is important to also have the weaker notion of \emph{soft} sheaf: 
a sheaf $\sh{F}$ is soft if all section of $\sh{F}$ above a \emph{closed} set can be extended to the ambiant space. 
This notion seems to have advantage to substitute the \emph{thin} sheaf, which played an essential role in the previous theory, as will be seen experimentally. 
On the paracompact space, if we have an exact sequence of soft sheaves of Abelian groups, the sections of these sheaves above any closed sets still formed an exact sequence.

\oldpage[III]
The paragraph 4 defines, for all space $X$, the same non-separated, and all sheaf $\sh{A}$ of Abelian groups on $X$, the group $\HH^n(X;\sh{A})$ and 
proof their essential properties: we have first of all
\[
    \HH^0(X;\sh{A}) = \Gamma({\sh{A}})
\]
group of global sections of $\sh{A}$; all have exact sequence
\[
    0\rightarrow \sh{A'} \rightarrow \sh{A} \rightarrow \sh{A''} \rightarrow 0
\]
is associated to a exact sequence of cohomology
\[
    \ldots\rightarrow \HH^n(X;\sh{A'}) \rightarrow \HH^n(X;\sh{A}) \rightarrow \HH^n(X;\sh{A''}) \rightarrow \HH^{n+1}(X;\sh{A'}) \ldots
\]
finally, we have
\[
    \HH^0(X;\sh{A}) = 0\ for\ n\geq 1
\]
if $\sh{A}$ is flabby (or, when $X$ is paracompact, if $\sh{A}$ is soft). 
The possibility of defintion, \emph{without any hypothesis of} $X$, of groups of cohomology possess these properties, 
have been proved first by ~A. ~Grothendick in 1955, using the fact of all sheaf can be extended into a \emph{injective} sheaf, in the sense of homological algebra. 
As the research of ~Grothendick on this subject will be published soon, we preferred to use, instead of injective sheaf, 
the flabby sheaf [which obviously are specially adapted to the study of functor $\sh{A}\rightarrow \Gamma({A})$]; 
it turns out that we can construct, in a canonical fashon, a \emph{flabby resolution}
\[
    0\rightarrow \sh{A} \rightarrow C^0(\sh{A}) \rightarrow C^1(\sh{A}) \rightarrow \ldots
\]
of all sheaf $\sh{A}$, which is also a ``exact'' functor with respect to the $\sh{A}$; posing
\[
    \cx{\CC}(X;\sh{A}) = \Gamma(\cx{C}(X;\sh{A})),\ 
    \HH^n(X;\sh{A}) = \HH^n(\cx{C}(X;\sh{A})),
\]
then we obtain, all the way to elementary results, the three fundemental properties of groups of cohomology.

The \sref[I]{I.4} also continue the proof, based on the usage of spectral sequence, the celebrated ``fundemental theorems''; 
In particular, all resolution
\[
    0\rightarrow \sh{A} \rightarrow \sh{L}^0 \rightarrow \sh{L}^1 \rightarrow \ldots
\]
of a sheaf $\sh{A}$ gives rise to the canonical homormorphisms
\[
    \HH^n(\Gamma(\cx{\sh{L}})) \rightarrow \HH^n(X;\sh{A})
\]
which are bijective if the $\sh{L}^p$ are flabby (or soft, when $X$ is paracompact), 
this obviously shows that in principle it was unneccessary, to choose a ``canonical'' flabby resolution of $\sh{A}$ to define the group $\HH^n(X;\sh{A})$.

\oldpage[IV]
Finally, we expose in detail the exact sequence of cohomology associated to a closed subspace, and 
various extra questions 
(relations between the cohomology of a set and its neighborhoods, cohomology value in an inductive limit of sheaves, spectral sequence of fiber spaces, cohomology dimension).

The \sref[I]{I.5} studies the relations between the groups $\HH^n(X;\sh{A})$ and the groups $\CHH^n(X;\sh{A})$ obtained by the method of \v{C}ech 
(which, we know it, does not give satisfying result outside of the paracompact space, or special categories of sheaves). 
We will show at first that all cover $U$ of $X$ that is open, is closed and locally finite, defines a \emph{resolution} $\cx{C}(U;\sh{A})$ of all sheaf $\sh{A}$ on $X$; 
It results to canonical homormorphisms $\CHH^n(X;\sh{A}) \rightarrow \HH^n(X;\sh{A})$ which comes from the residue of a spectral sequence. 

At the limit, we find a spectral sequence connecting the \v{C}ech cohomology to the ``good'' cohomology. 
We deduce that there is an isomorphism
\[
    \CHH^n(X;\sh{A}) = \HH^n(X;\sh{A})
\]
for all sheaf if $X$ is paracompact; 
and if $X$ is not paracompact, it is still the case for a given sheaf if there exists ``sufficent'' open sets in $X$ such that, for all finite intersection $U$ of all open sets, we have
\[
    \CHH^n(U;\sh{A}) = 0 \ for\ n\geq 1
\]
